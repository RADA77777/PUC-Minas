\documentclass[12pt]{article}
\usepackage{sbc-template}
\usepackage[utf8]{inputenc}
\usepackage{listings}
\usepackage{xcolor}
\usepackage{textcomp}
\setlength{\parindent}{0em}


\lstdefinestyle{style1}{
    basicstyle=\ttfamily,
    columns=fullflexible,
    keepspaces=true,
    upquote=true,
}
\lstdefinestyle{style2}{
    showstringspaces=false,
    commentstyle=\color{olive},
    keywordstyle=\color{blue},
    identifierstyle=\color{violet},
    stringstyle=\color{purple},
}
\lstdefinestyle{style3}{
    language=c,
    directivestyle=\color{teal},
}
\lstdefinestyle{combined}{
    style=style1,
    style=style2,
    style=style3,
}
\lstnewenvironment{code}{
    \lstset{
        style=combined,
    }
}{}


\title{Comparando DFS e Algoritmo de Prim para detecção de ciclos em grafos não-direcionados}
\author{Rafael Amauri Diniz Augusto}
\address{PUC-MG}

\begin{document}

\maketitle

\tableofcontents
\newpage

\section{Introdução}
O algoritmo de Prim e o DFS são algoritmos que têm várias aplicações na área de estudos de grafos. 
Uma dessas aplicações é o seu uso para detecção de ciclos em grafos não-direcionados conexos, o
que vai ser melhor explorado nesse artigo e como esses algoritmos podem ser alterados para
contar o número de ciclos em um grafo simples conexo não direcionado.\vspace{10pt}


\section{Explicando a classe Graph}


Nota: Todos os arquivos de código-fonte contém comentários e documentação extensiva. Favor
consultá-los para mais informações sobre o funcionamento do código.\vspace{10pt}


O trabalho da classe Graph é armazenar todas as informações sobre um determinado grafo, como 
a quais vértices um determinado vértice se conecta, qual o peso das arestas, o número de arestas
e o número de ciclos no grafo.\vspace{10pt}


\section{Explicando o Algoritmo de Prim}
O algoritmo de Prim é um algoritmo guloso de complexidade O(V²) - com V sendo o número de 
vértices - que tem como objetivo encontrar uma Árvore Geradora Mínima(AGM) para um determinado 
grafo. O seu funcionamento é escolhendo qual aresta dos vértices já visitados tem o menor valor, 
e em seguida adicionando ela a um novo grafo: a Árvore Geradora Mínima.

Para ser usado na contagem de ciclos de um determinado grafo, o funcionamento é bem simples: como
uma AGM é um grafo conexo sem ciclos, basta substrair o número de arestas da AGM do número de
arestas do grafo padrão. Esse valor será o número de ciclos no grafo.\vspace{10pt}


\section{Explicando o Algoritmo DFS}
O DFS é um algoritmo de travessia de grafos de complexidade O(V + E) - com V sendo o número de 
vértices e E sendo o número de arestas - se que originalmente começa a partir de um nó raiz e 
explora o grafo até onde for possível por meio de recursão.

Para ser usado na contagem de ciclos de um grafo, o DFS percorre o grafo marcando os vértices 
já visitados. Quando em um vértice V existe uma conexão para outro vértice já parcialmente
visitado, é porque existe um ciclo. Ao acumular quantas dessas conexões existem no grafo,
é encontrado o número de ciclos.\vspace{10pt}

\section{Comparação dos algoritmos e análise}

\end{document}